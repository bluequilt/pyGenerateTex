\documentclass[UTF8]{ctexart}
\usepackage{boxedminipage}
\usepackage{float}
\usepackage{hyperref}
\usepackage{graphicx}
\usepackage{array}
%================样式================
\usepackage[paperwidth=210mm,paperheight=297mm,top=25.4mm,bottom=12.7mm,left=12.7mm,right=12.7mm]{geometry}
%\usepackage{setspace}
%\setstretch{1}
\usepackage{enumitem}
\setenumerate[1]{partopsep=0mm,parsep=\parskip,topsep=0mm}
%================交叉引用================
\renewcommand{\figureautorefname}{图}
%================快捷================
\newcommand{\smalltitle}[1]{{\zihao{4}\bfseries{#1}}\\} 
%================页眉================
\usepackage{fancyhdr}
\usepackage{lastpage}
\pagestyle{fancy}
\renewcommand{\headrulewidth}{0.5pt}
%\renewcommand{\footrulewidth}{0pt}
\fancyhf{}
\setlength{\headheight}{35mm}
\chead{
\centering
{\zihao{3}\textbf{生产作业指导书}}\medskip\\
\zihao{5}
\setlength{\tabcolsep}{1mm}
\begin{tabular}{|m{32.24mm}|m{32.24mm}|m{32.24mm}|m{32.24mm}|m{43.65mm}|}
\hline
产品名称 & 产品件号 & 设备名称&设备编号&使用阶段\\
\hline
见\textless适用产品清单\textgreater&见\textless适用产品清单\textgreater&输入轴、扭杆压机&569-701077&wqert\\
\hline
工序号&工序名称&编制日期&版本号&页数\\
\hline
OP30/40& 蜗轮轴压扭杆/输入轴压自润滑衬套 &2020.5.14&B01&第 \thepage 页 \& 共 \pageref{LastPage} 页\\
\hline
\end{tabular}\\
}
%================正文的环境设置================
\begin{document}
\zihao{5}
\centering
%================正文内容================
\begin{boxedminipage}{184.6mm}
\centering
\smalltitle{开班准备描述}
\begin{boxedminipage}[t]{101.84mm}
\begin{enumerate}
\item 劳保鞋、静电服、静电手套穿戴整齐。不穿戴手套不得触摸零件。
\item 产线清线,非工作物料隔离。
\end{enumerate}
\end{boxedminipage}
\hfill
\begin{boxedminipage}[t]{78.76mm}
\begin{figure}[H]
\parbox[t]{37.38mm}{
\includegraphics[width=37.38mm]{pic01}
\caption{}}
\hfill
\parbox[t]{37.38mm}{
\includegraphics[width=37.38mm]{pic02}
\caption{}}
\end{figure}
\end{boxedminipage}
\end{boxedminipage}
\begin{boxedminipage}{184.6mm}
\centering
\smalltitle{设备开机描述}
\begin{boxedminipage}[t]{101.84mm}
\begin{enumerate}
\item 打开设备总电源,如图一。
\item 按下触摸屏下方的“设备上电”按钮,使其变为绿色,如图二。
\item 人员登录界面,选择“联机模式”,用户名输入“admin”,密码输入“admin”,点击“用户登录”,如图三。
\item 触摸屏点击上方的“目录页面”,随后在右侧选择“自动模式”如图四。
\item 此时触摸屏应当显示“没有报警”,即进入了正常生产模式。如若有报警,请按下触摸屏下方的“回初始位”按钮,设备回到原位后,该按钮应常亮黄色。随后转动“故障复位”旋钮,就能进入自动模式,如图五。
\end{enumerate}
\end{boxedminipage}
\hfill
\begin{boxedminipage}[t]{78.76mm}
图片2
\end{boxedminipage}
\end{boxedminipage}
\end{document}